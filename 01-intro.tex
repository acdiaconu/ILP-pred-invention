\section{Introduction}

\ac{something a bit more general about abstraction aimed at a wider AI audience. I will write this first paragraph.}

Inductive programming (IP) \cite{indprogmeets} - also known as program synthesis or example based learning - is a field that lies at the intersection of several computer science topics (machine learning, artificial intelligence, algorithm design) and is a form of automatic programming. IP tries to find a target program starting with an incomplete specification, which it tries to generalize into a program. Usually, that incomplete specification is represented by examples, so we can informally define inductive programming to be the process of creating programs from examples using a limited amount of background information. Inductive logic programming (ILP) represents IP in the context of logic programming, the target language usually being Prolog. For example, a problem commonly solved by ILP system is planning the route of a robot in a maze.

\ac{Paragraph 2: something about how PI is a form of abstraction ....}
\ac{give example}

\par Even in the early days of ILP, the idea of predicate invention (or discovery) has been thoroughly investigated, as a means to compensate for lack of knowledge. As noted by Russell \cite{humancompatible}, ``... this capability [discovery] is the most important step needed to reach human-level AI." Even though the idea of invention has been quite popular (and still is), an adjacent concept, which we shall call \emph{predicate reuse}, has not been given much attention. Informally, reuse is the ability of an ILP system to subsequently use a predicate, once that predicate has been invented. While the effectiveness of predicate invention has been documented, to the best of our knowledge there is no work that empirically demonstrates that predicate reuse is useful, nor any work discussing when it may be useful.
\rolf{Should note (somewhere) that determining the usefulness of invented predicates can be murky (which of Metagol's many IPs represent useful concepts?). By being used more than once, reused invented predicates do provide evidence for their usefulness.}

\ac{something about the limitations}

\ac{@SD what we do to address them}

Our contributions are:

\begin{itemize}
\item \ac{give a definition of reuse}

\item \ac{describe if/how ilp systems support PI}

\item \ac{describe if/how ilp systems support PI}Section 4 presents a wide range of experiments we have conducted, which highlight the benefits of using predicate invention in conjunction with predicate reuse. Section 5 discusses related work.
\end{itemize}
